\section{Parameterization Problem}

\subsection{Combining Mixin Classes}
The first solution we consider is adding new mixin class.
The full implementaion of this solution can be found in the university2 project.
The appraoch is to add a new mixin called \textit{EmailAndPhone} that call the print of the base object and of the email and phone mixin.
A new method is added to the existing mixins that only print email and phone respectivly.
The new method \textbf{emailPrint} of \textit{EmailAddress} can be seen in ~\coderef{EmailAddress2}.

\begin{lstlisting}[style=cpp, caption=\myCaption{}, label=emailandphone]
...
template <class Base>
class EmailAndPhone: public EmailAddress<Base>, public PhoneNumber<Base> //extends the mixins
{
public:
    EmailAndPhone(std::string fn,std::string ln, std::string t, std::string e, std::string pn)
        :EmailAddress(fn,ln,t,e),PhoneNumber(fn,ln,t,pn){}
        //The constructor calls the constructors of the other two mixins.
    
    const void print() 
    {
        Base::print();
        EmailAddress::emailPrint();
        PhoneNumber::phonePrint();
    }
};
...
\end{lstlisting}


\begin{lstlisting}[style=cpp, caption=\myCaption{}, label=EmailAddress2]
...
const void emailPrint() 
    {     
        cout << " " << _email;
    }
...
\end{lstlisting}

This apprach causes several problems. 
Firstly we have to define a new method in all our mixins to insure that they can be used alone and in combination. Secondly then we have to define a new mixin for each combination we want to use. And last then a instance of \textit{Base} in created both in \textit{EmailAddress} and \textit{PhoneNumber}, alternatively an additional constructor could be define for the mixin that do not require the parameters of the \textit{base} class. 
EmailAndPhone is the responsiable for calling the base class with the required parameters.

This naive approach require comprehensive rewriting of code and have at least three consequences. 

\subsection{Additional Constructors}


\subsection{Parameter Class}
