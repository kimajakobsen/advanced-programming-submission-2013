\section{Introduction}
A mixin (aka mixin class) provide ekstra functionality and or interface to other classes.
It is a piece of funktionality that normaly would be a part of a class but it have been sliced out shuch that it is no longer a part of the main class. 
It is however not ment for standalone use, but it should be reused by many different classes. 
The mixin is not suppost to be instantiated just like an abstract class. 
It is important to note that mixins is not the same as traits. 
A big difference between the two is that mixins contain states where traids is stateless. 
A mixin is often described as an abstract subclass. An example of a mixin is adding the top border to a window class. 
This mixin may be reused by several other window classes.
Porper use of mixins should yield less redundency and allow for reusable components which in the end improves the structure of the design.
Most compilers are able to process the mixins, except parameters, at compile time and thus reducing run-time costs. 
One of the biggest drawbacks when using mixins in that they suffers from complex parameterization.


In this report I explore the parameterization problem that occure when using mixins. 
This problem can be solved using structures such as Heterogeneous Value Lists as described in~\cite[p. 3]{drdobbs} or Mixin Layers as explained in~\cite{Smaragdakis:2000:MPC:645417.652070}. 
This repport only expore the problem and do not investergate the mentioned solutions. 
The history and development of mixins are explained in~\cite[Introduction]{Bracha:1990:MI:97946.97982}


In the next section an example of mixins in c++ is shown and explained. 
Thereafter I extend the code and demostrate the parameterization problem.
I then show three apporaches that futher underline the problem and the scalability problems that can occure when using mixins.
To widen our understanding of mixins I show an example of mixins in scala. 
The remaining part of the repport contains the conslusion and future work. 






